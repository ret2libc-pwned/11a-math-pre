\documentclass{beamer}

% If you use the optional handout parameter below, none of the pauses between slides are printed.  Only one of these first two lines should be used at a time.
% \documentclass[handout]{beamer}

\setbeamertemplate{navigation symbols}{}

\usepackage{amsmath, amsthm, amssymb, upgreek, amscd, verbatim, xcolor}
\usepackage{centernot}
\usepackage{datetime}
\usepackage{fontspec}
\usepackage[slantfont, boldfont]{xeCJK}
\usepackage{graphicx} 
\usepackage{float} 
\usepackage{subfigure}
\usepackage{tikz}
\usepackage{unicode-math} 
\usepackage{extarrows}
\usepackage{graphicx}


\usetheme{Madrid}
\usecolortheme{seahorse}
\setbeamertemplate{items}[circle]

% Use the XITS font
\setmathfont{XITSMath-Regular}
\setmainfont{XITSMath-Regular}
\setsansfont{XITSMath-Regular}
\setmonofont{XITSMath-Regular}

\setCJKmainfont{SimSun}[BoldFont=SimHei, ItalicFont=KaiTi]

% File matadata
\newcommand{\DocumentTitle}{浅谈几何分布}
\newcommand{\DocumentAuthor}{胡顾笑、张溥原、赵俣翔、邱义楷}

\hypersetup{
pdftitle={\DocumentTitle},
% pdfsubject={lorem ipsum},  % TODO
pdfauthor={\DocumentAuthor}, 
% pdfkeywords={lorem ipsum}  % TODO
}

%New Commands
\newcommand{\ssm}{\,{}^\smallsetminus\,}
\newcommand{\R}{\mathbb{R}}
\newcommand{\Q}{\mathbb{Q}}
\newcommand{\Z}{\mathbb{Z}}
\newcommand{\N}{\mathbb{N}}
\newcommand{\F}{\mathbb{F}}
\newcommand{\E}{\mathbb{E}}
\newcommand{\Prob}{\mathbb{P}}
\newcommand{\Geom}{\operatorname{Geom}}
\newcommand{\Var}{\operatorname{Var}}

\begin{document}
\title{\textbf\DocumentTitle}
\author[高二数学小组展示]{\DocumentAuthor}
\date{\today}

\maketitle

\begin{frame}{目录}
    \tableofcontents
\end{frame}

\section{引入问题}

\begin{frame}{例题}
    \begin{example}
        某饮料抽中“再来一瓶”的概率为 $1/3$,求期望喝到的瓶数。
    \end{example}
    \pause 
    \begin{solution}
        设期望喝到 $x$ 瓶,
        $$
        x = \frac 13(1 + x) + \frac23 (1)
        $$
        解得 $x = 3/2$。
    \end{solution}
\end{frame}


\section{几何分布}

\begin{frame}{定义}
    \begin{definition}
        回顾对 $n$ 重伯努利实验的定义:
        \begin{itemize}
            \item 结果只有两种:\textit{成功或失败}的实验,称为\textbf{伯努利实验}。
            \item 重复进行 $n$ 次\textit{相互独立}的伯努利实验,称为\textbf{ $n$ 重伯努利实验}。
        \end{itemize}
    \end{definition}
    \pause 
    \begin{definition}
        重复进行独立的伯努利实验 (每次实验的成功概率均为 $p$),直到第 $X$ 次成功为止。我们称 $X$ 服从\textbf{几何分布},或 $X \sim \Geom(p)$。
    \end{definition}
\end{frame}

\begin{frame}{分布、期望、方差}
    \begin{itemize}
        \item 分布
        \begin{theorem}
            若 $X\sim \Geom(p)$,则
            $$
                \Prob(X=k) = (1-p)^{k - 1}p
            $$
        \end{theorem}
        \pause
        \item 期望与方差
        \begin{corollary}
            \begin{itemize}
                \item $\E[X] = \frac1p$
                \item $\Var[X] = \frac{1 - p}{p^2}$
            \end{itemize}
        \end{corollary}
    \end{itemize}
\end{frame}

\begin{frame}[allowframebreaks]{对期望的证明}
    \begin{block}{Insight}
        根据期望的定义,有
        $$
            \begin{aligned}
                \E[X] &= \sum_{k = 1}^\infty k\Prob(X=k)\\
                &= p\sum_{k = 1}^\infty k(1-p)^{k - 1}
            \end{aligned}
        $$
        计算这个级数即可。
    \end{block}

    \begin{lemma}
        熟知,几何级数 (无穷等比数列的前 $n\to \infty$ 项和) 收敛仅当公比 $-1 < r < 1$,此时:
        $$
        \sum_{k = 1}^{\infty} r^k = \frac{1}{1 - r}
        $$
        % \pause
        注意到:上式两边求导后,
        $$
        \begin{aligned}
            \frac{\mathrm d}{\mathrm dr}\left(\sum_{k = 1}^{\infty} r^k\right) &=  \frac{\mathrm d}{\mathrm dr}\left(\frac{1}{1 - r}\right)\\
            \Rightarrow \sum_{k = 1}^{\infty} kr^{k - 1} &= \boxed{\frac{1}{(1 - r)^2}}
        \end{aligned}
        $$
    \end{lemma}

    \begin{proof}
        令上式中 $r = 1 - p$,
        $$
        \begin{aligned}
            \E[X] &= p\left[\frac{1}{(1 - 1 + p)^2}\right]
            &= \frac 1p
        \end{aligned}
        $$
    \end{proof}

\end{frame}

\section{彩蛋}

\begin{frame}{彩蛋:使用状态视角看``喝饮料''问题}
    \begin{definition}
        我们定义:
        \begin{itemize}
            \item 状态 $S$: 手上有一瓶饮料可以喝的状态
            \item 状态 0: 没有饮料可以喝了,游戏结束
        \end{itemize}
    \end{definition}
    \pause
    使用 $\E[state]$ 表示处于某个状态下喝到瓶数的期望,显然 $\E[0] = 0$。我们考虑状态间的转移:
    \pause
    \begin{proof}
    $$
        \begin{aligned}
            \E[S] &= \frac13 \E[S] + \frac23 \E[0] + 1 \\
            \Rightarrow \E[S] &= \frac32
        \end{aligned}
    $$
    \end{proof}
\end{frame}

\begin{frame}{彩蛋:几何分布和超几何分布}
    \begin{itemize}
        \item 几何分布的名称来源于几何级数
        \item 超几何分布的名称来源于超几何函数 (\href{https://en.wikipedia.org/wiki/Generalized_hypergeometric_function}{\color{blue}{Wikipedia}})
    \end{itemize}
\end{frame}

\section{致谢}

\begin{frame}{Thanks}
    \begin{center}
        {\Huge{谢谢!}}
    \end{center}
\end{frame}

\end{document}